\renewcommand{\baselinestretch}{1.25}%

\begin{table}[!t]
\caption{Generalized linear regressions of behavioral measures on losing in task 1, being paired with the messenger, and their interaction}
\begin{center}
\scalebox{1}{
\begin{threeparttable}
\begin{tabular}{l c c c}
\toprule
 & \makecell{Dictator\\Game} & \makecell{Spin the\\Wheel} & \makecell{Prisoner's\\Dilemma} \\
\midrule
Lost T1          & $-0.722$       & $-0.022$       & $0.107$        \\
                 & $(1.196)$      & $(0.127)$      & $(0.149)$      \\
Messenger        & $3.442^{**}$   & $0.928^{***}$  & $0.250$        \\
                 & $(1.122)$      & $(0.130)$      & $(0.148)$      \\
Lost x Messenger & $-6.450^{***}$ & $-1.085^{***}$ & $-0.747^{***}$ \\
                 & $(1.691)$      & $(0.184)$      & $(0.208)$      \\
Constant         & $13.083^{***}$ & $-0.102$       & $1.099^{***}$  \\
                 & $(0.845)$      & $(0.089)$      & $(0.102)$      \\
\midrule
Observations     & $1994$         & $1994$         & $1994$         \\
\bottomrule
\end{tabular}
\begin{tablenotes}[flushleft]
\scriptsize{\item[\hspace{-5mm}] \textit{Note:} Tobit regression with robust standard errors in parentheses for the first column (censoring at 0 and 50 cents),
                                and logistic regressions with robust standard errors in parentheses for the second and third column.
                                In the dictator game, the dependent variable (DV) is giving up to 50 cents to the partner. 
                                In the spin the wheel task, the DV is choosing the partner to spin the wheel on one’s behalf instead of the computer. 
                                In the prisoner’s dilemma, the DV is choosing cooperation. \item[\hspace{-5mm}] $^{***}p<0.001$; $^{**}p<0.01$; $^{*}p<0.05$.}
\end{tablenotes}
\end{threeparttable}
}
\label{tab:behavior_logit_regression}
\end{center}
\end{table}

\renewcommand{\baselinestretch}{1.67}%
