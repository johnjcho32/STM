\section{Methods}

\subsection{Study 1 Methods}

We recruited 2,762 participants from the United States on MTurk
\citep{berinsky2012evaluating}. 763 participants failed our preregistered
screeners (appendix), yielding a final sample of 1,999 subjects (61\%
female; age: \emph{M} = 41, \emph{SD} = 13 years) (see Table \ref{tab:exp1_sample} for more sample demographics). 
Participants read the
experimental instructions, answered comprehension questions, and
proceeded to the experiment. After the survey, participants were matched
with their partners and compensated based on the outcomes of tasks 1 and
2 as well as their performance in the quiz questions before each task.
The bonus was in addition to a flat fee for participating (70 cents).

\subsubsection{Design}

The experiment involved two parts. In the first part, treatment was
administered, as respondents were matched to a partner, a messenger, who
observed their performance in a task and communicated the result to
them. In the second part, participants played a dictator game, a
spin-the-wheel task, and a prisoner's dilemma game. In the first part,
participants were assigned to either a \emph{number counting task} or a
\emph{prediction task}. Specifically, they earned a bonus of 50 cents
for correctly identifying which of two 6 x 6 grids contained more odd
numbers in 20 seconds or for correctly predicting the outcome of a coin
toss. The number counting task was designed to give the impression that
the outcome is influenced by skill, whereas the prediction task aimed to
convey the idea that the outcome was the result of pure luck. However,
as intended, we did not find any statistical difference between the
outcomes of the counting and prediction tasks (other than the RTM effect for the prisoner's dilemma), 
meaning that empirically,
the counting task was equivalent to a coin toss. After the task,
participants received the news about the outcome from their partner
(``the messenger''). The messenger used either neutral language (``Your
count was correct/incorrect'' or ``Your coin toss prediction matched/did
not match the coin toss'') or emotional language (``Too bad! You lost!''
or ``Congratulations! You won!''). Then, participants read the
instructions for the second part of the experiment and took a
comprehension quiz.

After learning the outcome of the first part, subjects proceeded to the
second part. In the second part, respondents played three games with their
partner who was were randomly assigned to be either the messenger or a
new MTurk participant who was not the messenger. Specifically,
participants played a dictator game, a spin-the-wheel task, and a
prisoner's dilemma game in randomized order. They could earn
up to an extra 50 cents for each game depending on their own decisions
as well as the other player's. The dictator game split the total reward
between the respondent and the respondent. In the spin-the-wheel task,
they chose whether to have the computer or their partner (either the
messenger or a different MTurk participant) spin the wheel on their
behalf. Finally, the prisoner's dilemma was an up or down vote to
cooperate with the messenger or not.

In sum, participants were evenly assigned to one of eight conditions.
Specifically, we used a 2 (luck: prediction task vs. skill: number
counting task) x 2 (emotional language vs. neutral language) x 2
(messenger vs. another participant who is not the messenger)
between-subjects design.

Additionally, we randomly assigned one-third of the sample to answer
attitudes questions about the messenger just after the first task. The
remaining two-thirds of the sample was evenly split between participants
who answered the attitudes questions about the messenger or the partner
depending on whether they were paired with the messenger or the partner
in the three behavioral games. These questions were asked after the
games.

\subsubsection{Measures}

Our main pre-registered outcome measures are behavioral: participants'
decisions in the dictator game, spin-the-wheel task, and prisoner's
dilemma game. We analyzed participants' decisions in the games
separately, since each game aims to measure a different construct. The
dictator game measures altruism \citep{bekkers2007measuring, benenson2007children,johannesson2000non}, 
the spin-the-wheel task captures a
possible halo effect of the messenger, and the prisoner's dilemma is a
proxy for cooperation \citep{bendor1991doubt, rapoport1965prisoner}.\footnote{While 
we do not pre-register a combined index for the
  behavioral outcomes in Study 1 like we did in Study 2, we
  also created and analyzed a behavioral index variable combining these
  behavioral outcomes to be able to compare Studies 1 and 2.}

Additionally, we included attitudinal measures about the messenger and
non-messenger. To measure attitudes toward the messenger or the
non-messenger, participants rated their trustworthiness, niceness,
likeability, and generosity on a 7-point scale ranging from ``Not at
all'' to ``Extremely.'' The combined scale about attitudes toward the
messenger (analyses in the Appendix) (attitudes asked first: $\alpha =
0.91$; attitudes asked after: $\alpha = 0.95$) and the non-messenger were
reliable ($\alpha = 0.89$).\footnote{For the attitudes index variable, which combines these three scales,
we conduct principal component analysis on the four attitudes measures 
and find a sharp drop-off from the eigenvalue of the first dimension
to the second, thus validating our choice to use a single index.} At the end of the survey, participants had the
opportunity to provide open-ended comments.

\subsubsection{Research Design}

\begin{table}[]
\centering
\caption{Simplified research design of winning or losing compared to the messenger or non-messenger}
\label{tab:research-design}
\begin{threeparttable}
\begin{tabular}{@{}lcc@{}}
\toprule
\textbf{} & \multicolumn{1}{l}{\textbf{1st Round Win}} & \multicolumn{1}{l}{\textbf{1st Round Loss}} \\ \midrule
\textbf{Messenger} & A & B \\
\textbf{Non-messenger} & C & D \\ \bottomrule
\end{tabular}%
\begin{tablenotes}[flushleft]
\scriptsize{\item[\hspace{-5mm}] \textit{Note:} The columns represent whether the respondent received a message that said that they won or lost. The rows represent instances in which the respondent then plays a game with the messenger or non-messenger.}
\end{tablenotes}
\end{threeparttable}
\end{table}

For Study 1, we first look at a simplified research design that only
focuses on our main experimental conditions in Table \ref{tab:research-design} to determine the
quantities of interest. For each of the dependent variables (dictator
game, spin the wheel, prisoner's dilemma, behavioral index, trustworthy,
nice, likeable, generous, attitudes index), we calculate the same four
quantities of interest. First, the reward the messenger effect is
calculated by $A-C$, which asks whether the messenger gets rewarded
for delivering good news. Our prior hypothesis is that this RTM effect
is greater than 0. Second, the shoot the messenger effect is calculated
by $B-D$, which asks whether the messenger gets punished for
delivering bad news. Our prior hypothesis is that this STM effect is
less than 0. Finally, the total messenger bias (TMB), which asks whether
the subjects' behavior towards the messenger depends on delivering a
winning or losing message, is determined by whether both the STM and RTM
effect exists, which is calculated by $A-C = 0$ and $B-D =
0$.\footnote{As noted above, \citet{john2019shooting}'s analysis of the STM effect
 only looks at $A-B$, which may conflate the valence effect
  of receiving a bad news from actually shooting the messenger for
  delivering bad news. We avoid this with the inclusion of the
  non-messenger conditions when we measure the shoot the messenger
  effect.}

For the rest of our analyses, we use OLS linear regression to formally
analyze respondents' decisions. In each of the behavioral and attitudinal
outcomes, we model each outcome using linear regression with robust
standard errors and indicators for randomly assigned treatment
conditions.\footnote{As a robustness check, we also consider
more complex regression models for the behavioral measures. 
When we conduct generalized linear regressions (tobit regression with censoring at 0 and 50 cents for the dictator game variable, 
and logistic regression for the spin the wheel and prisoner’s dilemma variables)
in Tables \ref{tab:behavior_logit_regression}, 
\ref{tab:behavior_logit_regression_emotional}, and \ref{tab:behavior_logit_regression_counting}, following 
the logic of the main results, 
we do not find that substantive results vary from the OLS regressions for the behavioral measures.} 
That is, we specify:\footnote{To test whether the emotionality of the message or the type of task matters,
we interact the following regression with indicator variables for whether the respondent received an
emotional message and whether the respondent performed the counting task in Tables \ref{tab:behavior_regression_emotional}
and \ref{tab:behavior_regression_counting}, respectively. The latter regression is a deviation from our
pre-registered analysis from Study 1, in which we stated that we would examine the effect of the counting task
by interacting the 
type of task with the messenger instead.}

\[{Outcome}_{i} = \beta_{0} + \beta_{1}*Lost  + \beta_{2}*Messenger + \beta_{3}*Lost*Messenger + \varepsilon\]

Therefore, $H_0\colon \beta_2$ is an estimate of the effect
of rewarding the messenger, or RTM, (compared to rewarding the
non-messenger, the excluded category) while winning. Alternatively,
$H_o\colon \beta_2 + \beta_3$ is the estimate of the
effect of shooting the messenger, or STM, (compared to shooting the
non-messenger, the excluded category) while losing. We also investigate the relative magnitudes of rewarding and shooting the messenger, which
means we are testing whether the $RTM = -STM$ (since we expect the shoot
the messenger effect to be negative), which is a linear combination of
parameters test of $\beta_2 = -\beta_2 -
\beta_3$, which is equivalent to testing
$2\beta_2 + \beta_3 = 0$. Finally, the last quantity of interest is total
messenger bias (TMB), for which we conduct a joint hypothesis test that the RTM
and STM are simultaneously equal to 0, which is $H_0\colon \beta_2,
\beta_3 = 0$. 

\subsection{Study 2 Methods}

We recruited 3,576 participants from the United States on MTurk
\citep{berinsky2012evaluating}. 567 participants failed the initial screeners
and were excluded from proceeding further in the experiment, yielding a
final sample of 3,000 participants (60\% female; age: \emph{M} = 41,
\emph{SD} = 13 years) (see Table \ref{tab:exp2_sample} for more demographics). 
The flow of the experiment was the same as in
Study 1. After the survey, participants were matched and compensated
following the same procedure as used in Study 1. The bonus was in
addition to a flat fee for participating (70 cents). As in Study 1, we
recruited 99 messengers to ensure that participants' choices produced
real financial consequences for human messengers.

\subsubsection{Design}

The design was identical to Study 1 except in two respects. First, we
did not include emotional messages and the attitudes questions were
asked only after the behavioral games. Second, and more substantively,
we manipulated whether the messenger's fate was unrelated to the
participant, shared with the participant, or opposite of the
participant. Specifically, in the unrelated fate condition, the
messenger did not earn any additional money regardless of the outcome of
task 1. This was communicated in the message announcing the outcome,
where the messenger specified that they did not earn any money for
delivering the message. In contrast, when the messenger shared the same
fate as the participant, if a participant won 50 cents, the messenger
earned 5 cents or 50 cents depending on the condition. If a participant
lost and did not earn any money, the messenger did not earn any money
either. Similarly, in the opposite fate conditions, the messenger did
not earn any money when the participant won whereas they earned 5 or 50
cents depending on the condition when the participant lost in task 1. 
The messenger always announced their own earnings when they communicated
the outcome of task 1. In sum, in Study 2, we used a 2 (task 1:
prediction vs. count) x 2 (partner: messenger vs. another participant) x
5 (fate: unrelated vs. shared 5 vs. shared 50 vs. opposite 5 vs.
opposite 50) between-subjects design.

\subsubsection{Measures}

Our main outcome measures are the same as in Study 1, except that here
we preregistered that we would analyze the behavioral games as an index
of behavioral prosociality. The behavioral index is constructed as the
average of the standardized scores for the three behavioral games across
the entire sample. Each standardized score has \emph{M} = 0 and
\emph{SD} = 1. Thus, the resulting behavioral index has \emph{M} = 0,
\emph{SD} = 0.69, and ranges between $-1.3$ and $+1.3$, where a higher score
indicates more prosociality. The attitudes scale, which ranges from 1
(negative attitudes) to 7 (positive attitudes), had excellent
reliability ($\alpha = 0.94$). We pooled together the shared 5 and shared
50 cent fate conditions into one shared fate condition. 
Similarly, for the opposite 5 and 50 cent fate conditions, 
where the messenger earned money (5 or 50 cents
depending on the condition) when the participant lost and did not earn
any money when the participant won, we also pool these into one opposite fate 
condition.\footnote{In Figures
\ref{fig:study2_main_fivecent_behavior_all} and \ref{fig:study2_main_fivecent_attitude_all}, 
we break out each of the shared and
unrelated fate conditions to instances in which the partner received a 5 cent or
50 cent bonus. For the shared fate conditions, there are no differences between
the messenger winning or losing 50 cents or 5 cents for the RTM and STM effects
for both behaviors and attitudes (see Panels A and B). We see slight differences in the 
RTM effect between the 5 and 50 cent opposite fate conditions for both behaviors and attitudes. 
While the respondent continues to reward the messenger in the 5 cent opposite fate condition (Panel D), 
the RTM effect is not significant in the 50 cent opposite fate condition, in which
 the messenger wins 50 cents when the respondent
loses (Panel E).}

\subsubsection{Research Design}
In Study 2, we conduct the exact same analysis as Study 1, but focus on only the pooled
dependent variable (behavioral and attitudes index) for simplicity. We
analyze each of the alignment conditions (shared, unrelated, and
opposite fate) separately to see how the reward the messenger and shoot
the messenger effects shift when we specify the alignment of the
messenger.